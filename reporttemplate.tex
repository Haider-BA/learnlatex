%%%%%%%%%%%%%%%%%%%%%%%%%%%%%%%%%%%%%%
% Comments are started with the % sign
%
%         File: reporttemplate.tex
% Date Created: 2014 Mar 26
%  Last Change: 2014 Mar 26
%     Compiler: xelatex
%       Author: gil
%%%%%%%%%%%%%%%%%%%%%%%%%%%%%%%%%%%%%%

%%%%%%%%%%%%%%%%%%%%%%%%%%%%%%%%%%%%%%
%The first line of code specifies the document class and the font and
%paper size.  The article document class is a good place to start.  
%Some other options are:
%
%minimal - no fancy bits
%report - for a thesis or a longer report containing chapters
%book - for a book
%letter - nice for cover letters
%
%
%LaTex Libraries extend the functionality of the language.  
%The libraries used here are as follows:
%
%amsmath, amssymb - loading in math symbols and greek letters
%
%fontspec - lets you use fonts available on your system without manually
%installing them
%
%graphicx - for loading in figures/images
%
%booktabs, tabularx - for making pretty tables
%
%There are a lot of libraries available that can vastly extend LaTeX's
%abilities, but these will be sufficient for a solid report.
%
%%%
\documentclass[12pt,a4paper]{article}
\usepackage{amsmath, amssymb}
\usepackage{fontspec}
\usepackage{graphicx}
\usepackage{booktabs, tabularx}

\begin{document}
%Everything inside the \begin{document} and \end{document} tags will be
%compiled into our pdf

\section{The first section}
  We can write in plain text here and it will be typeset nicely.  If you
  look at the source code you'll see that the \verb|\section{}| command 
    has automatically made the section title a larger font and also numbered it.  

  \section*{The second section}
    By including an asterisk before the curly-braces, as in
  \verb|\section*{}|,  the automatic numbering is suppressed.  

  \subsection{Subsections}
  \subsubsection{and subsubsections}
  can be created using the \verb|\subsection{}| and
  \verb|\subsubsection{}| commands, respectively.  Again, if you add in
  an asterisk before the curly braces the automatic numbering will be
  suppressed. 

\section{Give up control to the machine}
  Don't worry about where your line breaks are.  \LaTeX\ ignores
  linebreaks in the source text.  
  So
  if
  I
  type
  a
  word
  on
  each
  line
  it
  will
  still
  be
  in
  a
  sentence.

  A full blank line needs to be used to start a new paragraph.  


  Two blank lines will still only make a new paragraph. So stop worrying
  about the formatting and just let \LaTeX\ handle it for you.  

\section{Typesetting math}
  Ok, now we really get started.  To insert an equation into a document,
  we have to tell \LaTeX\ that we want an equation environment.  In most
  helper programs there will be a shortcut to insert the appropriate
  lines.  

  \begin{equation}
    Bi = \frac{hr}{k}
    \label{eq:biotnumber}
  \end{equation}

  Notice that an equation number has been automatically inserted.  If we
  have a bunch of equations, we can rearrange them and they'll all be
  renumbered automatically.  In the source, there's also a line that
  reads \verb|\label{eq:biotnumber}| that isn't printed.  Those labels
  can be used to reference the appropriate equation.  If you want to
  draw attention to Equation (\ref{eq:biotnumber}), you can just
  reference the label.  Again, if you reorder your equations, the
  reference will automatically update.  
\end{document}
